\documentclass{article}
\usepackage{jheppub,physics,amsfonts,graphicx,tensor,cleveref}
\usepackage[export]{adjustbox}
\newcommand{\ii}{\mathrm{i}}
\newcommand{\me}{\mathrm{e}}
\newcommand{\cd}{\mathcal{D}}
\counterwithout{equation}{section}


\title{Homework 1}
\author{Shangjie Zhou}
\emailAdd{ZhouShangjie@purdue.edu}
\toccontinuoustrue
\notoc


\begin{document}

\maketitle


\section*{Problem 1}
For simplicity, we consider the case of swapping the $\alpha$-th and $\beta$-th electrons where $\alpha<\beta$.
The term $\exp(-\sum_{i=1}^Nz_i \bar{z}_i/4l^2)$ is obviously symmetric under $\alpha\leftrightarrow \beta$.
We only need to consider the terms involving $z_\alpha$ and $z_\beta$ in the term $\prod_{i<j}(z_i-z_j)^p$.

\begin{equation*}
    \prod_{j>\beta}(z_\alpha-z_j)\prod_{j>\beta}(z_\beta-z_j)
\end{equation*}


\begin{equation}
    \prod_{i<\alpha}(z_i-z_\alpha)\prod_{i<\beta}(z_i-z_\beta)
\end{equation}

\begin{equation}
    \prod_{i>\alpha}
\end{equation}

\section*{Problem 2}


\section*{Problem 3}
Consider the following Bogoliubov transformation:
\begin{subequations}\label{eq:bogoliubov}
    \begin{align}
        a&=ub+v^*b^\dagger\\
        a^\dagger&=u^*b^\dagger+vb
    \end{align}
\end{subequations}
where $u$ and $v$ are complex numbers satisfying $|u|^2-|v|^2=1$.
After plugging \cref{eq:bogoliubov} into the Hamiltonian, we have
\begin{equation}
    \begin{split}
        H&=E_0\left[(u^2+v^2)b^\dagger b+uv(b^2+b^{\dagger 2})+v^2\right]+E_1\left[(u^2+v^2)(b^{\dagger 2}+b^2)+4uvb^\dagger b+2uv\right]\\
         &= (E_0(u^2+v^2)+4E_1uv)b^\dagger b+(E_0uv+E_1(u^2+v^2))(b^2+b^{\dagger 2})+(E_0v^2+2E_1uv)
    \end{split}
\end{equation}

To eliminate terms $b^2$ and $b^{\dagger 2}$, we need to set
\begin{equation}
    E_0uv+E_1(u^2+v^2)=0
\end{equation}
We can parametrize $u$ and $v$ as
\begin{equation}
    u=\cosh\theta, \quad v=\sinh\theta
\end{equation}
Then the condition becomes
\begin{equation}
    E_0\cosh\theta\sinh\theta+E_1(\cosh^2\theta+\sinh^2\theta)=0
\end{equation}
and the solution is
\begin{equation}
    \tanh 2\theta=-\frac{2E_1}{E_0}.
\end{equation}
Therefore, the Hamiltonian becomes
\begin{equation}
    H=\sqrt{E_0^2-4E_1^2}b^\dagger b+\frac{1}{2}(\sqrt{E_0^2-4E_1^2}-E_0)
\end{equation}
We can see that the new Hamiltonian is a harmonic oscillator with frequency $\sqrt{E_0^2-4E_1^2}$, so we can write down the energy levels directly:
\begin{equation}
    E_n=n\sqrt{E_0^2-4E_1^2}+\frac{1}{2}(\sqrt{E_0^2-4E_1^2}-E_0)
\end{equation}
where $n=0,1,2,\ldots$.

\section*{Problem 4}
The time evolution of the particle density operator is given by
\begin{equation}
    \pdv{t}\rho(\vb{r},t)=\ii\comm{H}{\rho(\vb{r},t)}
\end{equation}
We can plug in the following Hamiltonian and density operator:
\begin{subequations}
    \begin{align}
        H&=H(t)=\sum_{\vb{k},\sigma}\epsilon_{\vb{k}}c_{\vb{k},\sigma}^\dagger c_{\vb{k},\sigma}\\
        \rho(\vb{r},t)&=\frac{1}{V}\sum_{\vb{k},\vb{q},\sigma}\me^{-\ii\vb{q}\cdot \vb{r}}c_{\vb{k}+\vb{q},\sigma}^\dagger(t) c_{\vb{k},\sigma}(t)
    \end{align}
\end{subequations}
Then we can compute
\begin{equation}
    \comm{H}{\rho(\vb{r},t)}=\frac{1}{V}\sum_{\vb{k},\vb{q},\vb{k}',\sigma,\sigma'}\me^{-\ii\vb{q}\cdot \vb{r}}\epsilon_{\vb{k}'}\comm{c_{\vb{k}',\sigma'}^\dagger(t) c_{\vb{k}',\sigma'}(t)}{c_{\vb{k}+\vb{q},\sigma}^\dagger(t) c_{\vb{k},\sigma}(t)}
\end{equation}
For the commutator, we have
\begin{equation}
    \begin{split}
        &\comm{c_{\vb{k}',\sigma'}^\dagger(t) c_{\vb{k}',\sigma'}(t)}{c_{\vb{k}+\vb{q},\sigma}^\dagger(t) c_{\vb{k},\sigma}(t)}\\
        =&c_{\vb{k}',\sigma'}^\dagger(t)\comm{c_{\vb{k}',\sigma'}(t)}{c_{\vb{k}+\vb{q},\sigma}^\dagger(t) c_{\vb{k},\sigma}(t)}+\comm{c_{\vb{k}',\sigma'}^\dagger(t)}{c_{\vb{k}+\vb{q},\sigma}^\dagger(t) c_{\vb{k},\sigma}(t)}c_{\vb{k}',\sigma'}(t)\\
        =&c_{\vb{k}',\sigma'}^\dagger(t)\acomm{c_{\vb{k}',\sigma'}(t)}{c_{\vb{k}+\vb{q},\sigma}^\dagger(t)}c_{\vb{k},\sigma}(t)-c_{\vb{k}+\vb{q},\sigma}^\dagger(t)\acomm{c_{\vb{k}',\sigma'}^\dagger(t)}{c_{\vb{k},\sigma}(t)}c_{\vb{k}',\sigma'}(t)\\
        =&\delta_{\vb{k}',\vb{k}+\vb{q}}\delta_{\sigma',\sigma}c_{\vb{k}',\sigma'}^\dagger(t)c_{\vb{k},\sigma}(t)-\delta_{\vb{k}',\vb{k}}\delta_{\sigma',\sigma}c_{\vb{k}+\vb{q},\sigma}^\dagger(t)c_{\vb{k}',\sigma'}(t)
    \end{split}
\end{equation}
and after plugging it back, we have
\begin{equation}\label{eq:rho-commutator}
    \begin{split}
        \pdv{t}\rho(\vb{r},t)&=\frac{\ii}{V}\sum_{\vb{k},\vb{q},\sigma}\me^{-\ii\vb{q}\cdot \vb{r}}(\epsilon_{\vb{k}+\vb{q}}-\epsilon_{\vb{k}})c_{\vb{k}+\vb{q},\sigma}^\dagger(t)c_{\vb{k},\sigma}(t)\\
                             &=\frac{\ii}{mV}\sum_{\vb{k},\vb{q},\sigma}\me^{-\ii\vb{q}\cdot \vb{r}}\left(\vb{k}+\frac{1}{2}\vb{q}\right)\cdot\vb{q}c_{\vb{k}+\vb{q},\sigma}^\dagger(t)c_{\vb{k},\sigma}(t)
    \end{split}
\end{equation}
where we have used free fermion energy $\epsilon_{\vb{k}}=\vb{k}^2/2m$ in the last step.

On the other hand, we also have the current density operator defined as
\begin{equation}
    \vb{j}(\vb{r},t)=\frac{1}{mV}\sum_{\vb{k},\vb{q},\sigma}\me^{-\ii\vb{q}\cdot \vb{r}}\left(\vb{k}+\frac{1}{2}\vb{q}\right)c_{\vb{k}+\vb{q},\sigma}^\dagger(t)c_{\vb{k},\sigma}(t)
\end{equation}
and its divergence is
\begin{equation}\label{eq:current-div}
    \div{\vb{j}(\vb{r},t)}=\frac{-\ii}{mV}\sum_{\vb{k},\vb{q},\sigma}\me^{-\ii\vb{q}\cdot \vb{r}}\left(\vb{k}+\frac{1}{2}\vb{q}\right)\cdot \vb{q}c_{\vb{k}+\vb{q},\sigma}^\dagger(t)c_{\vb{k},\sigma}(t)
\end{equation}

If we add \cref{eq:rho-commutator} and \cref{eq:current-div}, we can see that the continuity equation is satisfied:
\begin{equation}
    \pdv{t}\rho(\vb{r},t)+\div{\vb{j}(\vb{r},t)}=0
\end{equation}

%\bibliographystyle{jhep}
%\bibliography{ref}
\end{document}