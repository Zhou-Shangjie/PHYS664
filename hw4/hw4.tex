\documentclass{article}
\usepackage{jheppub,physics,amsfonts,graphicx,tensor,cleveref}
\usepackage[export]{adjustbox}
\newcommand{\ii}{\mathrm{i}}
\newcommand{\me}{\mathrm{e}}
\newcommand{\cd}{\mathcal{D}}
\DeclareMathOperator{\arcsinh}{arcsinh}
\DeclareMathOperator{\sgn}{sgn}
\counterwithout{equation}{section}


\title{Homework 4}
\author{Shangjie Zhou}
\emailAdd{zhou1468@purdue.edu}
\toccontinuoustrue
\notoc


\begin{document}

\maketitle


\section*{Problem 1}
\subsection*{(a)}
\subsubsection*{(i)}
From the commutation relation of the angular momentum operators, we have
\begin{subequations}
    \begin{align}
        S_z S_x-S_x S_z&=\ii \hbar S_y,\\
        S_y S_z-S_z S_y&=\ii \hbar S_x,
    \end{align}
\end{subequations}
which gives
\begin{subequations}
    \begin{align}
        \bra{S,S}(S_z S_x-S_x S_z)\ket{S,S}&=\ii \hbar \bra{S,S}S_y\ket{S,S},\\
        \bra{S,S}(S_y S_z-S_z S_y)\ket{S,S}&=\ii \hbar \bra{S,S}S_x\ket{S,S}.
    \end{align}
\end{subequations}
Since $S_z\ket{S,S}=\hbar S\ket{S,S}$, we have
\begin{subequations}
    \begin{align}
        \bra{S,S}(S_z S_x-S_x S_z)\ket{S,S}&=\hbar S \bra{S,S}S_x\ket{S,S}-\hbar S \bra{S,S}S_x\ket{S,S}=0,\\
        \bra{S,S}(S_y S_z-S_z S_y)\ket{S,S}&=\hbar S \bra{S,S}S_y\ket{S,S}-\hbar S \bra{S,S}S_y\ket{S,S}=0.
    \end{align}
    \end{subequations}
Thus, we have
\begin{subequations}
    \begin{align}
        \bra{S,S}S_y\ket{S,S}&=0,\\
        \bra{S,S}S_x\ket{S,S}&=0.
    \end{align}
\end{subequations}
\subsubsection*{(ii)}
Since $U=\exp(-\ii \phi S_z/\hbar)\exp(-\ii \theta S_y/\hbar)$, we can use Baker–Campbell–Hausdorff formula and compute that
\begin{equation}
    U^\dagger S_z U=\exp(\ii \theta S_y/\hbar)\,S_z\,\exp(-\ii \theta S_y/\hbar).
\end{equation}
Using the BCH formula, we have
\begin{equation}
    \begin{split}
        U^\dagger S_z U&= S_z +\frac{\ii \theta}{\hbar}[S_y,S_z] +\frac{1}{2!}\left(\frac{\ii \theta}{\hbar}\right)^2 [S_y,[S_y,S_z]] +\frac{1}{3!}\left(\frac{\ii \theta}{\hbar}\right)^3 [S_y,[S_y,[S_y,S_z]]]+\cdots\\
                       &= S_z +\frac{\ii \theta}{\hbar}(\ii \hbar S_x) +\frac{1}{2!}\left(\frac{\ii \theta}{\hbar}\right)^2 (\ii \hbar)(\ii \hbar S_z) +\frac{1}{3!}\left(\frac{\ii \theta}{\hbar}\right)^3 (\ii \hbar)(\ii \hbar)(\ii \hbar S_x)+\cdots\\
                       &= S_z \left(1 -\frac{\theta^2}{2!} +\frac{\theta^4}{4!} -\cdots\right) - S_x \left(\theta -\frac{\theta^3}{3!} +\frac{\theta^5}{5!} -\cdots\right)\\
                          &= S_z \cos\theta - S_x \sin\theta.
    \end{split}
\end{equation}


For $S_x$ we have a similar computation:
\begin{equation}
    \begin{split}
        U^\dagger S_x U&=\exp(\ii \theta S_y/\hbar)\, \exp(\ii \phi S_z/\hbar) S_x\,\exp(-\ii \phi S_z/\hbar)\exp(-\ii \theta S_y/\hbar)\\
                          &=\exp(\ii \theta S_y/\hbar) \left(S_x \cos\phi - S_y \sin\phi\right) \exp(-\ii \theta S_y/\hbar)\\
                            &= \big(\exp(\ii \theta S_y/\hbar) S_x \exp(-\ii \theta S_y/\hbar)\big)\cos\phi - S_y\sin\phi\\
                            &= \left(S_x\cos\theta + S_z\sin\theta\right)\cos\phi - S_y\sin\phi\\
                            &= S_x\cos\theta\cos\phi + S_z\sin\theta\cos\phi - S_y\sin\phi.
    \end{split}
\end{equation}

Similarly, for $S_y$ we have
\begin{equation}
    \begin{split}
        U^\dagger S_y U&=\exp(\ii \theta S_y/\hbar)\, \exp(\ii \phi S_z/\hbar) S_y\,\exp(-\ii \phi S_z/\hbar)\exp(-\ii \theta S_y/\hbar)\\
                          &=\exp(\ii \theta S_y/\hbar) \left(S_x \sin\phi + S_y \cos\phi\right) \exp(-\ii \theta S_y/\hbar)\\
                            &= \big(\exp(\ii \theta S_y/\hbar) S_x \exp(-\ii \theta S_y/\hbar)\big)\sin\phi + S_y\cos\phi\\
                            &= \left(S_x\cos\theta + S_z\sin\theta\right)\sin\phi + S_y\cos\phi\\
                            &= S_x\cos\theta\sin\phi + S_y\cos\phi + S_z\sin\theta\sin\phi.
    \end{split}
\end{equation}


\subsubsection*{(iii)}
Since $\vb{S}=S_x \vb{i}+S_y \vb{j}+S_z \vb{k}$ and $\vb{\omega}=\sin{\theta}\cos{\phi}\vb{i}+\sin{\theta}\sin{\phi}\vb{j}+\cos{\theta}\vb{k}$, we have
    \begin{equation}
    \begin{split}
        U^\dagger \vb{S} U&= U^\dagger S_x U \,\vb{i}+U^\dagger S_y U \,\vb{j}+U^\dagger S_z U \,\vb{k}\\
                          &= (S_x\cos\theta\cos\phi + S_z\sin\theta\cos\phi - S_y\sin\phi)\,\vb{i} \\
                          &\quad+ (S_x\cos\theta\sin\phi + S_y\cos\phi + S_z\sin\theta\sin\phi)\,\vb{j} \\
                          &\quad+ ( - S_x\sin\theta + S_z\cos\theta)\,\vb{k}\\
                          &= S_z \vb{\Omega}+\dots
    \end{split}
\end{equation}
we only keep the $S_z$ term since in the state $\ket{S,S}$ the expectation values of $S_x$ and $S_y$ are zero from part (i).

\subsubsection*{(iv)}
\begin{equation}
    \begin{split}
        \bra{\Omega}\vb{S}\ket{\Omega}=\bra{S,S}U^\dagger \vb{S} U\ket{S,S}= \bra{S,S} S_z \vb{\Omega} \ket{S,S}=\hbar S \vb{\Omega}.
    \end{split}
\end{equation}

\subsection*{(b)}
\subsubsection*{(i)}
Using $\ket{\omega}=\cos(\theta/2)\me^{-\ii\phi/2}\ket{1/2,1/2}+\sin(\theta/2)\me^{\ii\phi/2}\ket{1/2,-1/2}$ when $S=1/2$, we have the overlap
\begin{equation}
    \begin{split}
        \bra{\omega_2}\ket{\omega_1} &= \left(\cos\frac{\theta_2}{2}\me^{\ii\phi_2/2}\bra{1/2,1/2}+\sin\frac{\theta_2}{2}\me^{-\ii\phi_2/2}\bra{1/2,-1/2}\right)\\
        &\quad \cdot \left(\cos\frac{\theta_1}{2}\me^{-\ii\phi_1/2}\ket{1/2,1/2}+\sin\frac{\theta_1}{2}\me^{\ii\phi_1/2}\ket{1/2,-1/2}\right)\\
        &=\cos\frac{\theta_2}{2}\cos\frac{\theta_1}{2}\me^{\ii(\phi_2-\phi_1)/2}+\sin\frac{\theta_2}{2}\sin\frac{\theta_1}{2}\me^{-\ii(\phi_2-\phi_1)/2}.
    \end{split}
\end{equation}

\subsubsection*{(ii)}
For the representation of $\text{SU}(2)$ group, we have the following relation:
\begin{equation}\label{eq:su2}
    \underbrace{\frac{1}{2}\otimes\frac{1}{2}\otimes\dots\otimes\frac{1}{2}}_{2S \text{ times}}=S\oplus\text{lower spin representation}
\end{equation}
Since $\ket{S,S}$ is the highest weight state of the spin-$S$ representation, it must be a linear combination of states in the tensor product space with all individual spins being in their highest weight states:
\begin{equation}
    \ket{S,S}=\underbrace{\ket{1/2,1/2}\otimes\ket{1/2,1/2}\otimes\dots\otimes\ket{1/2,1/2}}_{2S \text{ times}}.
\end{equation}

\subsubsection*{(iii)}
From the \cref{eq:su2}, we know that 
\begin{equation}
    \begin{split}
        \ket{\Omega}_S=U_{S}\ket{S,S}&=\underbrace{U_{1/2}\ket{1/2,1/2}\otimes U_{1/2}\ket{1/2,1/2}\otimes\dots\otimes U_{1/2}\ket{1/2,1/2}}_{2S \text{ times}}\\
                                     &=\underbrace{\ket{\Omega}_{1/2}\otimes\ket{\Omega}_{1/2}\otimes\dots\otimes\ket{\Omega}_{1/2}}_{2S \text{ times}}.
    \end{split}
\end{equation}
where the underscript in $U$ indicates the representation it acts on.
Therefore, the overlap is
\begin{equation}
    \begin{split}
        \bra{\Omega_2}\ket{\Omega_1}_S&=\underbrace{\bra{\Omega_2}_{1/2}\otimes\bra{\Omega_2}_{1/2}\otimes\dots\otimes\bra{\Omega_2}_{1/2}}_{2S \text{ times}} \cdot \underbrace{\ket{\Omega_1}_{1/2}\otimes\ket{\Omega_1}_{1/2}\otimes\dots\otimes\ket{\Omega_1}_{1/2}}_{2S \text{ times}}\\
                                      &=\left(\bra{\Omega_2}_{1/2}\ket{\Omega_1}_{1/2}\right)^{2S}\\
                                      &=\left(\cos\frac{\theta_2}{2}\cos\frac{\theta_1}{2}\me^{\ii(\phi_2-\phi_1)/2}+\sin\frac{\theta_2}{2}\sin\frac{\theta_1}{2}\me^{-\ii(\phi_2-\phi_1)/2}\right)^{2S}.
    \end{split}
\end{equation}


\subsection*{(c)}
We can show that for $S=1/2$:
\begin{equation}
    \begin{split}
        \frac{2S+1}{4\pi}\int \dd{\Omega}\ket{\Omega}\bra{\Omega}&=\frac{2}{4\pi}\int_0^{2\pi}\dd{\phi}\int_0^{\pi}\dd{\theta}\sin{\theta}\left(\cos\frac{\theta}{2}\me^{-\ii\phi/2}\ket{1/2,1/2}+\sin\frac{\theta}{2}\me^{\ii\phi/2}\ket{1/2,-1/2}\right)\\
        &\quad \cdot \left(\cos\frac{\theta}{2}\me^{\ii\phi/2}\bra{1/2,1/2}+\sin\frac{\theta}{2}\me^{-\ii\phi/2}\bra{1/2,-1/2}\right)\\
        &=\int_0^{\pi}\dd{\theta}\sin{\theta}\left(\cos^2\frac{\theta}{2}\ket{1/2,1/2}\bra{1/2,1/2}+\sin^2\frac{\theta}{2}\ket{1/2,-1/2}\bra{1/2,-1/2}\right)
    \end{split}
\end{equation}
The remaining integrals can be computed as
\begin{subequations}
    \begin{align}
        \int_0^{\pi}\dd{\theta}\sin{\theta}\cos^2\frac{\theta}{2}&=\int_0^{\pi}\dd{\theta}\sin{\theta}\frac{1+\cos{\theta}}{2}=\frac{1}{2}\left(\int_0^{\pi}\dd{\theta}\sin{\theta}+\int_0^{\pi}\dd{\theta}\sin{\theta}\cos{\theta}\right)=1,\\
        \int_0^{\pi}\dd{\theta}\sin{\theta}\sin^2\frac{\theta}{2}&=\int_0^{\pi}\dd{\theta}\sin{\theta}\frac{1-\cos{\theta}}{2}=\frac{1}{2}\left(\int_0^{\pi}\dd{\theta}\sin{\theta}-\int_0^{\pi}\dd{\theta}\sin{\theta}\cos{\theta}\right)=1.
    \end{align}
\end{subequations}
Thus, we have
\begin{equation}
    \frac{2S+1}{4\pi}\int \dd{\Omega}\ket{\Omega}\bra{\Omega}=\ket{1/2,1/2}\bra{1/2,1/2}+\ket{1/2,-1/2}\bra{1/2,-1/2}=\mathbb{I}_{1/2}.
\end{equation}

Using the resolution of identity, the propagator can be written as
\begin{equation}
    \begin{split}
        K(\Omega_f,t_f;\Omega_i,t_i)&=\bra{\Omega_f}\me^{-\ii \hat{H}(t_f-t_i)/\hbar}\ket{\Omega_i}\\
                                    &=\left(\frac{2S+1}{4\pi}\right)^{N-1}\int \dd{\Omega_1}\dd{\Omega_2}\dots\dd{\Omega_{N-1}}\\
        &\quad \cdot \bra{\Omega_f}\me^{-\ii \hat{H}(t_f-t_i)/N\hbar}\ket{\Omega_{N-1}}\bra{\Omega_{N-1}}\me^{-\ii \hat{H}(t_f-t_i)/N\hbar}\ket{\Omega_{N-2}}\dots\\
        &\quad \cdot \bra{\Omega_1}\me^{-\ii \hat{H}(t_f-t_i)/N\hbar}\ket{\Omega_i},
    \end{split}
\end{equation}

\subsection*{(d)}
We can expand the matrix element as
\begin{equation}
    \begin{split}
        \mel**{\Omega_{k+1}}{\me^{-\ii \hat{H}\epsilon/\hbar}}{\Omega_k}&=\mel**{\Omega_{k+1}}{\mathbb{I}-\frac{\ii \epsilon}{\hbar}\hat{H}+\mathcal{O}(\epsilon^2)}{\Omega_k}\\
        &=\bra{\Omega_{k+1}}\ket{\Omega_k}-\frac{\ii \epsilon}{\hbar}\bra{\Omega_{k+1}}\hat{H}\ket{\Omega_k}+\mathcal{O}(\epsilon^2)\\
        &=\bra{\Omega_{k+1}}\ket{\Omega_k}\left(1-\frac{\ii \epsilon}{\hbar}\frac{\bra{\Omega_{k+1}}\hat{H}\ket{\Omega_k}}{\bra{\Omega_{k+1}}\ket{\Omega_k}}+\mathcal{O}(\epsilon^2)\right)\\
        &=\bra{\Omega_{k+1}}\ket{\Omega_k}\exp\left(-\frac{\ii \epsilon}{\hbar}\frac{\bra{\Omega_{k+1}}\hat{H}\ket{\Omega_k}}{\bra{\Omega_{k+1}}\ket{\Omega_k}}\right)\\
        &=\exp(\ln{\bra{\Omega_{k+1}}\ket{\Omega_k}}-\frac{\ii \epsilon}{\hbar}\frac{\bra{\Omega_{k+1}}\hat{H}\ket{\Omega_k}}{\bra{\Omega_{k+1}}\ket{\Omega_k}})\\
        &=\exp(\ln{\bra{\Omega_{k+1}}\ket{\Omega_k}}-\frac{\ii\epsilon}{\hbar}\mathcal{H}(\vb{\Omega_k}))
    \end{split}
\end{equation}

If we write $\theta_{k+1}=\theta_k+\dot{\theta}_k\epsilon$ and $\phi_{k+1}=\phi_k+\dot{\phi}_k\epsilon$, then to first order in $\epsilon$ the correct expansion is
\begin{equation}
    \begin{split}
        \ln{\bra{\Omega_{k+1}}\ket{\Omega_k}}&=2S\ln\Big(\cos\tfrac{\theta_{k+1}}{2}\cos\tfrac{\theta_k}{2}\,\me^{\ii(\phi_{k+1}-\phi_k)/2}+\sin\tfrac{\theta_{k+1}}{2}\sin\tfrac{\theta_k}{2}\,\me^{-\ii(\phi_{k+1}-\phi_k)/2}\Big)\\
        &=2S\ln\Big(1+\ii\tfrac{\dot{\phi}_k\epsilon}{2}\cos\theta_k\Big)\\
        &=2S\Big(\ii\tfrac{\dot{\phi}_k\epsilon}{2}\Big)\cos\theta_k\\
        &=\ii S\dot{\phi}_k\epsilon\cos\theta_k.
    \end{split}
\end{equation}
Therefore, the propagator can be written as
\begin{equation}
    \begin{split}
        K(\Omega_f,t_f;\Omega_i,t_i)&=\lim_{N\to\infty}\left(\frac{2S+1}{4\pi}\right)^{N-1}\int \dd{\Omega_1}\dd{\Omega_2}\dots\dd{\Omega_{N-1}}\\
        &\quad \cdot \exp\Bigg(\sum_{k=0}^{N-1}\left[\ii S\dot{\phi}_k\epsilon\cos\theta_k -\frac{\ii\epsilon}{\hbar}\mathcal{H}(\vb{\Omega_k})\right]\Bigg)\\
        &=\int \cd{(\Omega(t))}\exp\left(\frac{\ii}{\hbar}\int_{t_i}^{t_f}\dd{t}\left[\hbar S \dot{\phi}\cos\theta-\mathcal{H}(\vb{\Omega}(t))\right]\right)\\
        &=\int_{\Omega(t_i)=\Omega_i}^{\Omega(t_f)=\Omega_f}\cd\Omega(t)\me^{\frac{\ii}{\hbar}S[\vb{\Omega}]}
    \end{split}
\end{equation}
where the action is
\begin{equation}
    S[\vb{\Omega}]=\int_{t_i}^{t_f}\dd{t}\left[\hbar S \dot{\phi}(1-\cos\theta)-\mathcal{H}(\vb{\Omega}(t))\right].
\end{equation}
and we have added a total time derivative term $\hbar S \dot{\phi}$ to the Lagrangian which does not affect the equations of motion.

\subsection*{(e)}

\section*{Problem 2}
\subsection*{(a)}

%\bibliographystyle{jhep}
%\bibliography{ref}
\end{document}