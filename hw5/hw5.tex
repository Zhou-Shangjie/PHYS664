\documentclass{article}
\usepackage{jheppub,physics,amsfonts,graphicx,tensor,cleveref}
\usepackage[export]{adjustbox}
\newcommand{\ii}{\mathrm{i}}
\newcommand{\me}{\mathrm{e}}
\newcommand{\cd}{\mathcal{D}}
\DeclareMathOperator{\arcsinh}{arcsinh}
\DeclareMathOperator{\sgn}{sgn}
\counterwithout{equation}{section}


\title{Homework 5}
\author{Shangjie Zhou}
\emailAdd{zhou1468@purdue.edu}
\toccontinuoustrue
\notoc


\begin{document}

\maketitle


\section*{Problem 1}


\section*{Problem 2}


\section*{Problem 3}
\subsection*{(i)}
\subsubsection*{(a)}
We can directly verify that 
\begin{equation}
    (\mu^x_{i+1/2})^2=\sigma^z_i \sigma^z_{i+1} \sigma^z_i \sigma^z_{i+1}=(\sigma^z_i)^2(\sigma^z_{i+1})^2=1
\end{equation}
\begin{equation}
    (\mu^z_{i+1/2})^2=\prod_{j\leq i} \sigma^x_j \prod_{k\leq i} \sigma^x_k= \prod_{j\leq i} (\sigma^x_j)^2=1
\end{equation}
\begin{equation}
    \acomm{\mu^x_{i+1/2}}{\mu^z_{i+1/2}}=\sigma^z_i \sigma^z_{i+1} \prod_{j\leq i} \sigma^x_j + \prod_{j\leq i} \sigma^x_j \sigma^z_i \sigma^z_{i+1}=\acomm{\sigma^z_i}{\sigma^x_i} \sigma^z_{i+1} \prod_{j<i} \sigma^x_j=0
\end{equation}
For $i+1/2\neq j+1/2$, we have
\begin{equation}
    \comm{\mu^z_{i+1/2}}{\mu^z_{j+1/2}}=\comm{\prod_{k\leq i}\sigma^x_k}{\prod_{k\leq j}\sigma^x_k}=0
\end{equation}
For $\abs{i-j}>1$, obviously we have
\begin{equation}
    \comm{\mu^x_{i+1/2}}{\mu^x_{j+1/2}}=\comm{\sigma^z_i \sigma^z_{i+1}}{\sigma^z_j \sigma^z_{j+1}}=0
\end{equation}
\begin{equation}
    \comm{\mu^x_{i+1/2}}{\mu^z_{j+1/2}}=\comm{\sigma^z_i \sigma^z_{i+1}}{\prod_{k\leq j}\sigma^x_k}=0
\end{equation}

\subsection*{(ii)}

\section*{Problem 4}
\subsection*{(a)}
If we use the following transformation:
\begin{equation}
    S_i^+=c_i^\dagger, S_i^-=c_i, S_i^z=c_i^\dagger c_i-\frac{1}{2}
\end{equation}
then the commutation relations of spin operators on the same site can be verified as:
\begin{subequations}
    \begin{align}
        \comm{S_i^+}{S_i^-}&=\comm{c_i^\dagger}{c_i}= 2c_i^\dagger c_i-1=2S_i^z\\
        \comm{S_i^z}{S_i^+}&=\comm{c_i^\dagger c_i-\frac{1}{2}}{c_i^\dagger}=c_i^\dagger=\ S_i^+\\
        \comm{S_i^z}{S_i^-}&=\comm{c_i^\dagger c_i-\frac{1}{2}}{c_i}=-c_i=-S_i^-
    \end{align}
\end{subequations}
For spins at different sites $i\neq j$, the commutation relation can be computed by using $\acomm{c_i^\dagger}{c_j}=0$:
\begin{subequations}
    \begin{align}
        \comm{S_i^+}{S_j^-}&=\comm{c_i^\dagger}{c_j}=2c_i^\dagger c_j\neq 0\\
        \comm{S_i^z}{S_j^+}&=\comm{c_i^\dagger c_i-\frac{1}{2}}{c_j^\dagger}=0\\
        \comm{S_i^z}{S_j^-}&=\comm{c_i^\dagger c_i-\frac{1}{2}}{c_j}=0
    \end{align}
\end{subequations}
So the transformation doesn't preserve the commutation relations between spins at different sites.

\subsection*{(b)}
If we use the following transformation:
\begin{equation}
    S_i^+=c_i^\dagger \me^{\ii \pi \sum_{k<i} c_k^\dagger c_k},\quad S_i^-= \me^{-\ii \pi \sum_{k<i} c_k^\dagger c_k} c_i,\quad S_i^z=c_i^\dagger c_i-\frac{1}{2}
\end{equation}
then the commutation relations of spin operators on the same site can be verified as:
\begin{subequations}
    \begin{align}
        \comm{S_i^+}{S_i^-}&= \me^{\ii \pi \sum_{k<i} c_k^\dagger c_k} \comm{c_i^\dagger}{c_i} \me^{-\ii \pi \sum_{k<i} c_k^\dagger c_k}=2c_i^\dagger c_i-1=2S_i^z\\
        \comm{S_i^z}{S_i^+}&=\comm{c_i^\dagger c_i-\frac{1}{2}}{c_i^\dagger} \me^{\ii \pi \sum_{k<i} c_k^\dagger c_k}=c_i^\dagger \me^{\ii \pi \sum_{k<i} c_k^\dagger c_k}=\ S_i^+\\
        \comm{S_i^z}{S_i^-}&=\me^{-\ii \pi \sum_{k<i} c_k^\dagger c_k}\comm{c_i^\dagger c_i-\frac{1}{2}}{c_i}=-c_i \me^{-\ii \pi \sum_{k<i} c_k^\dagger c_k}=-S_i^-
    \end{align}
\end{subequations}
For spins at different sites $i< j$, the commutation relation can be computed by using $\acomm{c_i^\dagger}{c_j}=0$:
\begin{equation}
    \begin{split}
        \comm{S_i^+}{S_j^-}&=\comm{c_i^\dagger \me^{\ii\pi\sum_{k<i}c_k^\dagger c_k}}{\me^{-\ii\pi\sum_{k<j}c_k^\dagger c_k} c_j}\\
                           &=c_i^\dagger \me^{-\ii\pi\sum_{i\leq k<j}c_k^\dagger c_k} c_j-\me^{-\ii\pi\sum_{i\leq k<j}c_k^\dagger c_k} c_j c_i^\dagger\\
                           &=\comm{c_i^\dagger}{c_j \me^{-\ii\pi\sum_{i\leq k<j}c_k^\dagger c_k}}\\
                           &=
    \end{split}
\end{equation}
this is because when $i<j$, the exponential factor $\me^{-\ii\pi\sum_{i\leq k<j}c_k^\dagger c_k}$ contains $c_i$ and $c_i^\dagger$, which anticommute with $c_j$ and $c_j^\dagger$. 


\subsection*{(c)}

%\bibliographystyle{jhep}
%\bibliography{ref}
\end{document}