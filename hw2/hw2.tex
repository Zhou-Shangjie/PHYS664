\documentclass{article}
\usepackage{jheppub,physics,amsfonts,graphicx,tensor,cleveref}
\usepackage[export]{adjustbox}
\newcommand{\ii}{\mathrm{i}}
\newcommand{\me}{\mathrm{e}}
\newcommand{\cd}{\mathcal{D}}
\DeclareMathOperator{\arcsinh}{arcsinh}
\DeclareMathOperator{\sgn}{sgn}
\counterwithout{equation}{section}


\title{Homework 2}
\author{Shangjie Zhou}
\emailAdd{ZhouShangjie@purdue.edu}
\toccontinuoustrue
\notoc


\begin{document}

\maketitle


\section*{Problem 1}
\subsection*{(a)}
When $t>t'$,
\begin{equation}
    \theta(t-t')G^{>}(t-t')+\theta(t'-t)G^{<}(t-t')=G^{>}(t-t')=-\ii\expval{c(t)c^{\dagger}(t')}.
\end{equation}
When $t<t'$,
\begin{equation}
    \theta(t-t')G^{>}(t-t')+\theta(t'-t)G^{<}(t-t')=G^{<}(t-t')=\ii\expval{c^{\dagger}(t')c(t)}.
\end{equation}
Therefore, we have 
\begin{equation}
    G^{T}(t-t')=\theta(t-t')G^{>}(t-t')+\theta(t'-t)G^{<}(t-t').
\end{equation}

\subsection*{(b)}
We first write down the Fourier transform of $G^{>}(t-t')$ and $G^{<}(t-t')$:
\begin{subequations}
    \begin{align}
        G^{>}(t-t')&=\int_{-\infty}^{+\infty}\frac{\dd{\omega}}{2\pi}G^{>}(\omega)\me^{-\ii\omega(t-t')},\\
        G^{<}(t-t')&=\int_{-\infty}^{+\infty}\frac{\dd{\omega}}{2\pi}G^{<}(\omega)\me^{-\ii\omega(t-t')}.
    \end{align}
\end{subequations}
Then the Fourier transform of $G^{T}(t-t')$ is
\begin{equation}
    \begin{split}
        G^{T}(\omega)&=\int_{-\infty}^{+\infty}\dd{t}G^{T}(t)\me^{\ii\omega(t)}\\
        &=\int_{-\infty}^{+\infty}\dd{t}\qty[\theta(t)G^{>}(t)+\theta(-t)G^{<}(t)]\me^{\ii\omega t}\\
        &=\int_{-\infty}^{+\infty}\dd{t}\qty[\theta(t)\int_{-\infty}^{+\infty}\frac{\dd{\omega'}}{2\pi}G^{>}(\omega')\me^{-\ii\omega' t}+\theta(-t)\int_{-\infty}^{+\infty}\frac{\dd{\omega'}}{2\pi}G^{<}(\omega')\me^{-\ii\omega' t}]\me^{\ii\omega t}\\
        &=\int_{-\infty}^{+\infty}\frac{\dd{\omega'}}{2\pi}G^{>}(\omega')\int_{0}^{+\infty}\dd{t}\me^{\ii(\omega-\omega')t}+\int_{-\infty}^{+\infty}\frac{\dd{\omega'}}{2\pi}G^{<}(\omega')\int_{-\infty}^{0}\dd{t}\me^{\ii(\omega-\omega')t}\\
        &=\int_{-\infty}^{+\infty}\frac{\dd{\omega'}}{2\pi\ii}\qty[-\frac{G^{>}(\omega')}{\omega-\omega'+\ii0^{+}}+\frac{ G^{<}(\omega')}{\omega-\omega'-\ii0^{+}}]
    \end{split}
\end{equation}

\subsection*{(c)}
We will use the following identity of the spectral function when temperature is zero:
\begin{subequations}
    \begin{align}
        G^{>}(\omega)&=-2\pi\ii A(\omega)\theta(\omega)\\
        G^{<}(\omega)&=2\pi\ii A(\omega)\theta(-\omega)\\
        G^{R}(\omega)&= \int_{-\infty}^{+\infty}\dd{\omega'}\frac{A(\omega)}{\omega-\omega'+\ii0^{+}}\\
        G^{A}(\omega)&= \int_{-\infty}^{+\infty}\dd{\omega'}\frac{A(\omega)}{\omega-\omega'-\ii0^{+}}
    \end{align}
\end{subequations}

Now $G^{T}(\omega)$ can be written as
\begin{equation}
    G^{T}(\omega)=\int_{-\infty}^{+\infty}\dd{\omega'}\frac{A(\omega')}{\omega-\omega'+\sgn(\Im(\omega))\ii 0^+}
\end{equation}

When $\omega$ is on the upper half plane, we have 
\begin{equation}
    G^{T}(\omega)=\int_{-\infty}^{+\infty}\dd{\omega'}\frac{A(\omega)}{\omega-\omega'+\ii0^{+}}=G^R(\omega)
\end{equation}
in the last step we used the fact the imaginary part of $\omega$ is positive.
Using the same argument, when $\omega$ is on the lower half plane, we have $G^{T}(\omega)=G^{A}(\omega)$.

\subsection*{(d)}
For the anti-time-ordered Green's function, we have
\begin{equation}
    G^{\tilde{T}}(t-t')=\theta(t'-t)G^{>}(t-t')+\theta(t-t')G^{<}(t-t')
\end{equation}
and its Fourier transform can be written in terms of the spectral function as
\begin{equation}
    G^{\tilde{T}}(\omega)=\int_{-\infty}^{+\infty}\frac{\dd{\omega'}}{2\pi\ii}\qty[-\frac{G^<(\omega')}{\omega-\omega'+\ii0^{+}}+\frac{G^>(\omega')}{\omega-\omega'-\ii0^{+}}]=-\int_{-\infty}^{+\infty}\dd{\omega'}\frac{A(\omega')}{\omega-\omega'+\sgn(\Im(\omega))\ii 0^+}.
\end{equation}
so when $\omega$ is on the upper half plane, we have $G^{\tilde{T}}(\omega)=-G^R(\omega)$, and when $\omega$ is on the lower half plane, we have $G^{\tilde{T}}(\omega)=-G^A(\omega)$.


\section*{Problem 2}
\subsection*{(a)}
To get the imaginary part of the retarded Green's function, we can rewrite it as 
\begin{equation}
    G^{R}(\vb{r},\vb{r}';\omega)=\sum_{\alpha}\frac{\phi_\alpha(\vb{r})\phi^\ast_\alpha(\vb{r}')}{\omega-E_\alpha+\ii 0^+}=\lim_{\epsilon\to 0^+}\sum_{\alpha}\frac{\phi_\alpha(\vb{r})\phi^\ast_\alpha(\vb{r}')}{\omega-E_\alpha+\ii\epsilon}=\lim_{\epsilon\to 0^+}\sum_\alpha \frac{\phi_\alpha(\vb{r})\phi_\alpha^*(\vb{r'})(\omega-E_\alpha-\ii \epsilon)}{(\omega-E_\alpha)^2+\epsilon^2}.
\end{equation}
Now we can see that the imaginary part is
\begin{equation}
    \Im G^{R}(\vb{r},\vb{r}';\omega)=\sum_\alpha \frac{-\ii \epsilon\phi_\alpha(\vb{r})\phi_\alpha^*(\vb{r'})}{(\omega-E_\alpha)^2+\epsilon^2}.
\end{equation}
So the imaginary part of the retarded Green's function is related to the eigenmodes and eigenenergies.

\subsection*{(b)}
Using the identity 
\begin{equation}
    \frac{1}{x\pm \ii 0^+}=\mathcal{P}\frac{1}{x}\mp \ii \pi \delta(x),
\end{equation}
the retarded Green's function can be rewritten as
\begin{equation}
    G^{R}(\vb{r},\vb{r}';\omega)=\sum_{\alpha}\phi_\alpha(\vb{r})\phi_\alpha^*(\vb{r'})\qty[\mathcal{P}\frac{1}{\omega-E_\alpha}-\ii \pi \delta(\omega-E_\alpha)].
\end{equation}
and the imaginary part is
\begin{equation}
    \Im G^{R}(\vb{r},\vb{r}';\omega)=-\pi \sum_{\alpha}\phi_\alpha(\vb{r})\phi_\alpha^*(\vb{r'})\delta(\omega-E_\alpha)
\end{equation}
and we can see that 
\begin{equation}
    -\frac{1}{\pi}\Im G^{R}(\vb{r},\vb{r};\omega)=\sum_{\alpha}\abs{\phi_\alpha(\vb{r})}^2\delta(\omega-E_\alpha)=\rho(\vb{r},\omega)
\end{equation}

\subsection*{(c)}
After integrating over $\vb{r}$, we have
\begin{equation}
    \rho(\omega)=\int \dd{\vb{r}}\rho(\vb{r},\omega)=-\frac{1}{\pi}\int \dd{\vb{r}}\Im G^{R}(\vb{r},\vb{r};\omega)=-\frac{1}{\pi}\Im \Tr G^{R}(\omega).
\end{equation}
where $\int \dd{\vb{r}}G^{R}(\vb{r},\vb{r};\omega)=\Tr G^{R}(\omega)$.

\subsection*{(d)}
For a free particle in $d$ dimensions, the retarded Green's function can be expressed as
\begin{equation}
    G^{R}(\vb{r},\vb{r}';\omega)=\int \frac{\dd{\vb{k}}}{(2\pi)^d}\frac{e^{i\vb{k}\cdot(\vb{r}-\vb{r'})}}{\omega-E_{\vb{k}}+\ii 0^+},
\end{equation}
where $E_{\vb{k}}=\frac{\hbar^2 k^2}{2m}$ is the energy dispersion relation for a free particle.
The trace of the retarded Green's function is
\begin{equation}
    \Tr G^{R}(\omega)=\int \dd{\vb{r}}G^{R}(\vb{r},\vb{r};\omega)=\int \dd{\vb{r}}\int \frac{\dd{\vb{k}}}{(2\pi)^d}\frac{1}{\omega-E_{\vb{k}}+\ii 0^+}=V\int \frac{\dd{\vb{k}}}{(2\pi)^d}\frac{1}{\omega-E_{\vb{k}}+\ii 0^+},
\end{equation}
where $V$ is the volume of the system.
Now we can calculate the density of states:
\begin{equation}
    \rho(\omega)=-\frac{1}{\pi}\Im \Tr G^{R}(\omega)=-\frac{V}{\pi}\Im \int \frac{\dd{\vb{k}}}{(2\pi)^d}\frac{1}{\omega-E_{\vb{k}}+\ii 0^+}=\frac{V}{\pi}\int \frac{\dd{\vb{k}}}{(2\pi)^d}\pi \delta(\omega-E_{\vb{k}}).
\end{equation}
By using the volume of the $d$-dimensional sphere:
\begin{equation}
    \int \dd{\Omega_d}=\frac{2\pi^{d/2}}{\Gamma(d/2)}.
\end{equation}
the above integral can be evaluated as
\begin{equation}
    \rho(\omega)=\frac{V}{(2\pi)^d}\frac{2\pi^{d/2}}{\Gamma(d/2)}\int_0^{+\infty} k^{d-1}\dd{k}\delta(\omega-E_{\vb{k}}).
\end{equation}
Using the property of the delta function, we have
\begin{equation}
    \delta(\omega-E_{\vb{k}})=\delta\qty(\omega-\frac{\hbar^2 k^2}{2m})=\frac{m}{\hbar^2 k}\delta\qty(k-\sqrt{\frac{2m\omega}{\hbar^2}}).
\end{equation}
So the density of states is
\begin{equation}
    \rho(\omega)=\frac{V}{(2\pi)^d}\frac{2\pi^{d/2}}{\Gamma(d/2)}\int_0^{+\infty} k^{d-1}\dd{k}\frac{m}{\hbar^2 k}\delta\qty(k-\sqrt{\frac{2m\omega}{\hbar^2}})=\frac{V}{(2\pi)^d}\frac{2\pi^{d/2}}{\Gamma(d/2)}\frac{m}{\hbar^2}\qty(\sqrt{\frac{2m\omega}{\hbar^2}})^{d-2}
\end{equation}
which can be simplified to
\begin{equation}
    \rho(\omega)=V\qty(\frac{m}{2\pi \hbar^2})^{d/2}\frac{1}{\Gamma(d/2)}\omega^{d/2-1}.
\end{equation}

On the other hand, the density of states can also be calculated by counting the number of states in a volume of $k$-space:
\begin{equation}
    \rho(\omega)=\dv{N}{\omega}=\dv{N}{k}\dv{k}{\omega}.
\end{equation}
The number of states in a volume of $k$-space is
\begin{equation}
    N=\frac{V}{(2\pi)^d}\int \dd{\Omega_d}\int_0^k k'^{d-1}\dd{k'}=\frac{V}{(2\pi)^d}\frac{2\pi^{d/2}}{\Gamma(d/2)}\frac{k^d}{d}.
\end{equation}
where $V/(2\pi)^d$ is the density of states in $k$-space.
So we have
\begin{equation}
    \dv{N}{k}=\frac{V}{(2\pi)^d}\frac{2\pi^{d/2}}{\Gamma(d/2)}k^{d-1}.
\end{equation}
The derivative of $k$ with respect to $\omega$ is
\begin{equation}
    \dv{k}{\omega}=\dv{}{\omega}\sqrt{\frac{2m\omega}{\hbar^2}}=\sqrt{\frac{m}{2\hbar^2}}\omega^{-1/2}.
\end{equation}
Therefore, the density of states is
\begin{equation}
    \rho(\omega)=\frac{V}{(2\pi)^d}\frac{2\pi^{d/2}}{\Gamma(d/2)}k^{d-1}\sqrt{\frac{m}{2\hbar^2}}\omega^{-1/2}=\frac{V}{(2\pi)^d}\frac{2\pi^{d/2}}{\Gamma(d/2)}\qty(\sqrt{\frac{2m\omega}{\hbar^2}})^{d-1}\sqrt{\frac{m}{2\hbar^2}}\omega^{-1/2}
\end{equation}
which can be simplified to
\begin{equation}
    \rho(\omega)=V\qty(\frac{m}{2\pi \hbar^2})^{d/2}\frac{1}{\Gamma(d/2)}\omega^{d/2-1}.
\end{equation}
We can see that the density of states calculated by the two methods are consistent.



\section*{Problem 3}
\subsection*{(a)}
We will first calculate the following commutator:
\begin{equation}
    \begin{split}
        \comm{d}{H}=&\sum_{k}\epsilon_k \comm{d}{c_k^\dagger c_k}+\epsilon_d \comm{d}{d^\dagger d}+V\sum_k \comm{d}{c_k^\dagger d}+V\sum_k \comm{d}{d^\dagger c_k}\\
    \end{split}
\end{equation}
Notice that 
\begin{equation}
    \comm{d}{c_k^\dagger c_k}=d c_k^\dagger c_k - c_k^\dagger c_k d=d c_k^\dagger c_k-d c_k^\dagger c_k=0,
\end{equation}
\begin{equation}
    \comm{d}{d^\dagger d}=d d^\dagger d - d^\dagger d d=dd^\dagger d=d\acomm{d^\dagger}{d}-ddd^\dagger=d,
\end{equation}
\begin{equation}
    \comm{d}{c_k^\dagger d}=d c_k^\dagger d - c_k^\dagger d d=0,
\end{equation}
\begin{equation}
    \comm{d}{d^\dagger c_k}=d d^\dagger c_k - d^\dagger c_k d=dd^\dagger c_k + d^\dagger d c_k=\acomm{d}{d^\dagger}c_k=c_k.
\end{equation}
so we have
\begin{equation}
    \comm{d}{H}=\epsilon_d d + V\sum_k c_k.
\end{equation}

Similarly, we can calculate the commutator $\comm{c_k}{H}$:
\begin{equation}
    \comm{c_k}{H}=\epsilon_k c_k + V d.
\end{equation}

Therefore, the EOMs are 
\begin{subequations}
    \begin{align}
        \ii \dv{t}d(t)&=\comm{d(t)}{H}=\epsilon_d d(t) + V\sum_k c_k(t),\\
        \ii \dv{t}c_k(t)&=\comm{c_k(t)}{H}=\epsilon_k c_k(t) + V d(t).
    \end{align}
\end{subequations}

If we write a time ordered Green's function as
\begin{equation}
    \expval{T A(t)B(0)}=\theta(t)\expval{A(t)B(0)}- \theta(-t)\expval{B(0)A(t)},
\end{equation}
then its time derivative is
\begin{equation}
    \begin{split}
        \dv{t}\expval{T A(t)B(0)}&=\delta(t)\qty(\expval{A(t)B(0)}+\expval{B(0)A(t)})+\theta(t)\expval{\dv{t}A(t)B(0)}-\theta(-t)\expval{B(0)\dv{t}A(t)}\\
                                 &=\delta(t)\expval{\acomm{A(t)}{B(0)}}+\expval{T\dv{A(t)}{t}B(0)}\\
                                 &=\delta(t)\expval{\acomm{A(0)}{B(0)}}+\expval{T\dv{A(t)}{t}B(0)}
    \end{split}
\end{equation}

For the time ordered Green's function, we have the following EOM after differentiation:
\begin{equation}
    \begin{split}\label{eq:3a1}
        \dv{t}G_d(t)&=-\ii\delta(t)-\ii\expval{T \qty(\dv{t}d(t))d^\dagger(0)}\\
                    &=-\ii\delta(t)-\expval{T(\epsilon_d d(t) + V\sum_k c_k(t))d^\dagger(0)}\\
                    &=-\ii\delta(t)-\ii \epsilon_d G_d(t)-\ii V\sum_k G_{ck}(t)
    \end{split}
\end{equation}

\begin{equation}
    \begin{split}\label{eq:3a2}
        \dv{t}G_{c_k,d^\dagger}(t)&=-\ii\expval{T \qty(\dv{t}c_k(t))d^\dagger(0)}\\
                                  &=-\expval{T(\epsilon_k c_k(t) + V d(t))d^\dagger(0)}\\
                                  &=-\ii \epsilon_k G_{c_k,d^\dagger}(t)-\ii V G_d(t)
    \end{split}
\end{equation}
After substituting the Fourier transform of the Green's functions, \cref{eq:3a1,eq:3a2} become
\begin{equation}
    \omega G_d(\omega)=1+\epsilon_d G_d(\omega)+V\sum_k G_{c_k,d^\dagger}(\omega)
\end{equation}
\begin{equation}
    \omega G_{c_k,d^\dagger}(\omega)=\epsilon_k G_{c_k,d^\dagger}(\omega)+V G_d(\omega)
\end{equation}
Now $G_d(\omega)$ can be solved:
\begin{equation}
    G_d(\omega)=\frac{1}{\omega-\epsilon_d-V^2\sum_k \frac{1}{\omega-\epsilon_k}}=\frac{G_{d,0}(\omega)}{1-G_{d,0}(\omega)\Delta(\omega)},
\end{equation}
where $G_{d,0}(\omega)=\frac{1}{\omega-\epsilon_d}$ and $\Delta(\omega)=V^2\sum_k \frac{1}{\omega-\epsilon_k}$.
So $G_d(\omega)$ satisfies a Dyson-like equation.

\subsection*{(b)}
We can write the self-energy as
\begin{equation}
    \Sigma(\omega)=V^2\sum_k \frac{1}{\omega-\epsilon_k+ \ii 0^+}=\int \dd{k}\frac{V^2 \rho(\epsilon_k)}{\omega-\epsilon_k+ \ii 0^+}.
\end{equation}
To extract the imaginary part of the self-energy, we can use the identity
\begin{equation}
    \frac{1}{x\pm \ii 0^+}=\mathcal{P}\frac{1}{x}\mp \ii \pi \delta(x),
\end{equation}
so we have
\begin{equation}
    \Sigma(\omega)=\int \dd{k}\frac{V^2 \rho(\epsilon_k)}{\omega-\epsilon_k+ \ii 0^+}=\int \dd{k}\frac{V^2 \rho(\epsilon_k)}{\omega-\epsilon_k}-\ii \pi V^2 \rho(\omega),
\end{equation}
and in the wide-band limite $\rho(\epsilon)=\rho_0$, we have
\begin{equation}
    \Delta(\omega)\simeq -\ii\Gamma,
\end{equation}
up to a constant real part and $\Gamma=\pi V^2 \rho_0$.

\subsection*{(c)}
In the wide-band limit, the impurity Green's function is
\begin{equation}
    G^R_d(\omega)=\frac{1}{\omega-\epsilon_d+\ii \Gamma}.
\end{equation}
which can be rewritten as 
\begin{equation}
    G^R_d(\omega)=\frac{\omega-\epsilon_d-\ii \Gamma}{(\omega-\epsilon_d)^2+\Gamma^2}.
\end{equation}
and we can compute the spectral function: 
\begin{equation}
    A_d(\omega)=-\frac{1}{\pi}\Im G^R_d(\omega)=\frac{1}{\pi}\frac{\Gamma}{(\omega-\epsilon_d)^2+\Gamma^2}.
\end{equation}

\subsection*{(d)}
To verify that $A_d(\omega)$ reduces to a Dirac delta function, we can compute the following integral:
\begin{equation}
    \int_{-\infty}^{+\infty} A_d(\omega)\dd{\omega}=\int_{-\infty}^{+\infty}\frac{1}{\pi}\frac{\Gamma}{(\omega-\epsilon_d)^2+\Gamma^2}\dd{\omega}=\frac{1}{\pi}\qty[\arctan\frac{\omega-\epsilon_d}{\Gamma}]_{-\infty}^{+\infty}=1. 
\end{equation}

\section*{Problem 4}
\begin{equation}
    \begin{split}
        W&=\sum_{\alpha,\beta}\expval{T_\tau c_\alpha^\dagger(\tau_1) c_\beta^\dagger(\tau_1)c_\alpha(\tau_2)c_\beta(\tau_2)}\\
         &=\expval{T_\tau c_1^\dagger(\tau_1) c_2^\dagger(\tau_1)c_1(\tau_2)c_2(\tau_2)}+\expval{T_\tau c_2^\dagger(\tau_1) c_1^\dagger(\tau_1)c_2(\tau_2)c_1(\tau_2)}\\
         &=2\expval{T_\tau c_1^\dagger(\tau_1) c_2^\dagger(\tau_1)c_1(\tau_2)c_2(\tau_2)}
    \end{split}
\end{equation}

From the Hamiltonian $H=\sum_{\alpha=1}^2\epsilon_\alpha c_\alpha^\dagger c_\alpha+U n_1 n_2=\sum_{\alpha=1}^2 \epsilon_\alpha n_\alpha+U n_1 n_2$, we have the energy eigenstates and eigenvalues:
\begin{itemize}
    \item $\ket{0,0}$, $E_0=0$
    \item $\ket{1,0}=c_1^\dagger\ket{0,0}$, $E_{1,0}=\epsilon_1$
    \item $\ket{0,1}=c_2^\dagger\ket{0,0}$, $E_{0,1}=\epsilon_2$
    \item $\ket{1,1}=c_1^\dagger c_2^\dagger\ket{0,0}$, $E_{1,1}=\epsilon_1+\epsilon_2+U$
\end{itemize}
\subsection*{Exact Evaluation}
\begin{equation}
    \begin{split}
        &\expval{T_\tau c_1^\dagger(\tau_1) c_2^\dagger(\tau_1)c_1(\tau_2)c_2(\tau_2)}\\
        &=\theta(\tau_1-\tau_2)\expval{c_1^\dagger(\tau_1) c_2^\dagger(\tau_1)c_1(\tau_2)c_2(\tau_2)}-\theta(\tau_2-\tau_1)\expval{c_1(\tau_2)c_2(\tau_2)c_1^\dagger(\tau_1) c_2^\dagger(\tau_1)}                                                
    \end{split}
\end{equation}

So we can compute 
\begin{equation}
    \begin{split}
        \expval{c_1^\dagger(\tau_1) c_2^\dagger(\tau_1)c_1(\tau_2)c_2(\tau_2)}&=\frac{1}{Z}\Tr\qty[e^{-\beta H}c_1^\dagger(\tau_1) c_2^\dagger(\tau_1)c_1(\tau_2)c_2(\tau_2)]\\
                                                                              &=\frac{1}{Z}\Tr\qty[e^{-\beta H}e^{\tau_1 H}c_1^\dagger c_2^\dagger e^{-\tau_1 H}e^{\tau_2 H}c_1 c_2 e^{-\tau_2 H}]\\
                                                                              &=\frac{1}{Z}\Tr\qty[e^{-(\beta-\tau_1+\tau_2) H}c_1^\dagger c_2^\dagger e^{-(\tau_1-\tau_2) H}c_1 c_2]\\
                                                                              &=\frac{1}{Z}\bra{1,1}e^{-(\beta-\tau_1+\tau_2) H}c_1^\dagger c_2^\dagger e^{-(\tau_1-\tau_2) H}c_1 c_2\ket{1,1}\\
                                                                              &=\frac{1}{Z}e^{-(\beta-\tau_1+\tau_2) E_{1,1}}e^{-(\tau_1-\tau_2) E_{0,0}}\\
    \end{split}
\end{equation}
\begin{equation}
    \expval{c_1(\tau_2)c_2(\tau_2)c_1^\dagger(\tau_1) c_2^\dagger(\tau_1)}=\frac{1}{Z}e^{-(\beta-\tau_2+\tau_1) E_{0,0}}e^{-(\tau_2-\tau_1) E_{1,1}}
\end{equation}

Therefore, we have 
\begin{equation}
    \begin{split}
        W&=\frac{2}{Z}\qty[\theta(\tau_1-\tau_2)e^{-(\beta-\tau_1+\tau_2) E_{1,1}}e^{-(\tau_1-\tau_2) E_{0,0}}-\theta(\tau_2-\tau_1)e^{-(\beta-\tau_2+\tau_1) E_{0,0}}e^{-(\tau_2-\tau_1) E_{1,1}}]\\
            &=\frac{2}{Z}\qty[\theta(\tau_1-\tau_2)e^{-(\beta-\tau_1+\tau_2) (\epsilon_1+\epsilon_2+U)}-\theta(\tau_2-\tau_1)e^{-(\tau_2-\tau_1) (\epsilon_1+\epsilon_2+U)}]
    \end{split}
\end{equation}

\subsection*{Wick's Theorem}
Using Wick's theorem, we have
\begin{equation}
    \expval{T_\tau c_1^\dagger(\tau_1) c_2^\dagger(\tau_1)c_1(\tau_2)c_2(\tau_2)}=-\expval{T_\tau c_1^\dagger(\tau_1)c_1(\tau_2)}\expval{T_\tau c_2^\dagger(\tau_1)c_2(\tau_2)}
\end{equation}
and 
\begin{equation}
    \expval{T_\tau c_1^\dagger(\tau_1)c_1(\tau_2)}=\theta(\tau_1-\tau_2)\expval{c_1^\dagger(\tau_1)c_1(\tau_2)}-\theta(\tau_2-\tau_1)\expval{c_1(\tau_2)c_1^\dagger(\tau_1)}
\end{equation}
and 
\begin{equation}
    \expval{c_1^\dagger(\tau_1)c_1(\tau_2)}=\frac{1}{Z}\Tr\qty[e^{-\beta H}c_1^\dagger(\tau_1)c_1(\tau_2)]=\frac{1}{Z}e^{-(\beta-\tau_1+\tau_2)\epsilon_1}
\end{equation}
and 
\begin{equation}
    \expval{c_1(\tau_2)c_1^\dagger(\tau_1)}=\frac{1}{Z}\Tr\qty[e^{-\beta H}c_1(\tau_2)c_1^\dagger(\tau_1)]=\frac{1}{Z}e^{-(\tau_2-\tau_1)\epsilon_1}
\end{equation}
Therefore, we have 
\begin{equation}
    W=\frac{2}{Z^2}\qty[\theta(\tau_1-\tau_2)e^{-(\beta-\tau_1+\tau_2) (\epsilon_1+\epsilon_2)}-\theta(\tau_2-\tau_1)e^{-(\tau_2-\tau_1) (\epsilon_1+\epsilon_2)}]
\end{equation}

It is clear that the result from Wick's theorem is different from the exact evaluation.


%\bibliographystyle{jhep}
%\bibliography{ref}
\end{document}